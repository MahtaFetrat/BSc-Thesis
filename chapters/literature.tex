
\فصل{کارهای پیشین}

کارهای پیشین انجام شده در حوزه‌ی ASPECTS از نظر روش مورد استفاده، در چند دسته‌ی کلی قابل بررسی هستند.
در طی بررسی هر دسته، ابتدا روش کلی مورد استفاده در آن توضیح داده می‌شود. 
سپس به نمونه‌هایی از کارهای پیشین که در آن چهارچوب کار کرده‌اند اشاره می‌شود
و نتایج به‌دست آمده توسط این کار‌ها عنوان شده و مورد مقایسه قرار می‌گیرد.

\قسمت{روش ناحیه‌بندی
\footnote{Segmentation}
 و طبقه‌بندی
 \footnote{Classification}
 }
در این روش، ده بخش مورد توجه ASPECTS در تصاویر مغزی ناحیه‌بندی
 می‌شوند.
به این ترتیب،  
مدل یادگیری ماشین به طور مستقیم از محل این نواحی در تصاویر آگاهی می‌یابد.
سپس مدل آموزش داده می‌شود که هر ناحیه‌ای که می‌بیند، آیا آسیب دیده‌است یا خیر.
یعنی یاد می‌گیرد که هر ناحیه را به دو دسته‌ی آسیب‌دیده و سالم طبقه‌بندی
کند.
در نهایت، ، سپس با جمع امتیازات تمام ده ناحیه‌ی هر بیمار، امتیاز ASPECTS وی به دست می‌آید.\\

\subsection{ناحیه‌بندی نواحی ASPECTS}
ناحیه‌بندی ۱۰ بخش ASPECTS تصاویر به دو طریق مختلف انجام می شود.
روش اول از یادگیری ماشین بهره می‌گیرد.
در این روش، هر تصویر مغزی، برچسبی دارد که نشان می‌دهد کدام پیکسل‌های تصویر متعلق به هر ناحیه هستند.
تعداد زیادی از تصاویر مغزی به همراه این برچسب‌ها به مدل ورودی داده تا ناحیه‌بندی را بیاموزد.
به این ترتیب، مدل می‌تواند با دریافت یک تصویر مغزی جدید و بدون برچسب، مشخص کند که کدام پیکسل‌ها متعلق به هر ناحیه هستند.\\

روش دیگر ناحیه‌بندی، مبتنی بر یادگیری نیست و نیازی به تعداد زیادی تصویر به همراه برچسب ندارد.
بلکه در این روش، یک یا چند تصویر مغزی استاندارد، به عنوان الگو
\footnote{Template}
، برچسب زده می‌شوند.
سپس به کمک روش‌های انطباق تصاویر
\footnote{Image registration}
، تصویر الگو بر یک تصویر مغزی مورد نظر منطبق می‌شود تا نواحی مشخص شده روی آن، در تصویر جدید هم مشخص شوند.
از جمله روش‌های منطبق کردن تصویر الگو بر روی تصویر جدید، جابجایی، دوران، بزرگ‌نمایی، تغییر شکل جزئی و \dots می‌باشد.
تصویر الگو آن‌قدر دچار این دست تغییرات می‌شود تا معیار شباهتش با تصویر جدید، به حد مطلوبی برسد.
یک نمونه‌ی ساده از چنین معیاری می‌تواند مجموع اختلاف قدر مطلق دو تصویر باشد که باید کمینه شود.
لازم به ذکر است که روش‌های ناحیه‌بندی به کمک انطباق تصاویر، عموما توانایی کمتری نسبت به مدل‌های 
یادگیری ماشین دارند اما نسبت به آن روش‌ها نیازمندی‌های داده‌ای کمتری دارند.\\

\subsection{استخراج ویژگی نواحی}
پس از مشخص شدن محدوده‌ی هر ناحیه‌ی ،ASPECTS لازم است ویژگی‌های اصلی هر ناحیه استخراج شود تا مدل بتواند از روی این ویژگی‌ها، آن ناحیه را دسته‌بندی کند.
در کارهای پیشین، محاسبه‌ی چنین ویژگی‌هایی به دو طریق مختلف انجام شده‌است.
دسته‌ی اول، استخراج ویژگی‌های هر ناحیه را به مدل یادگیری ماشین واگذار می‌کنند.
یعنی تصاویر به مدل، ورودی داده می‌شوند و مدل طی چندین مرحله مشاهده‌ی نواحی به همراه برچسبشان، می‌آموزد که چه ویژگی‌هایی از تصاویر استخراج کند که بیش از همه مفید باشند.\\

اما دسته‌ی دیگر برای استخراج ویژگی تصاویر، به جای یادگیری ماشین، روش‌های محاسباتی و پردازش تصویری را به کار می‌گیرند.
در واقع یک‌سری ویژگی‌های آماری همچون میانگین و واریانس شدت رنگ پیکسل‌ها برای هر ناحیه محاسبه می‌شوند.
پس از اینکه این ویژگی‌ها برای هر ناحیه استخراج شدند، 
در اختیار مدل یادگیری ماشین یا هوش مصنوعی قرار می‌گیرند
تا 
در طبقه‌بندی نواحی ، استفاده شوند.

\subsection{نمونه‌ی کارهای پیشین}
یکی از تازه‌ترین پژوهش‌ها در زمینه‌ی امتیاز‌دهی خودکار ،ASPECTS در همین دسته‌ از روش‌ها قرار می‌گیرد.
%Clinical evaluation of a deep-learning model for automatic scoring of the Alberta stroke program early CT score on non-contrast CT
این پژوهش با ناحیه‌بندی نواحی ASPECTS و استخراج ویژگی‌های نواحی به کمک مدل یادگیری ماشین، توانسته به دقت های نسبتا خوبی 
(تشخیص
\footnote{specificity}
۶۳.۹۶٪
و
حساسیت
\footnote{sensitivity}
۷۸.۶۲٪
در امتیازدهی ده‌گانه و 
تشخیص
۵۶.۷۶٪
و
حساسیت
۴۲.۹۵٪
در امتیازدهی دوبخشی 
)
دست یابد.
نمونه‌ی موفق و اخیر دیگری وجود دارد که نواحی را به کمک یادگیری ماشین ناحیه‌بندی و طبقه‌بندی می‌کند.
%Deep learning derived automated ASPECTS on non-contrast CT scans of acute ischemic stroke patients
این نمونه نیز نتایج بسیار خوبی 
(تشخیص
۲.۹۲٪
و
حساسیت
۲.۷۷٪
در امتیازدهی ده‌گانه و 
تشخیص
۶.۸۶٪
و
حساسیت
۵.۹۵
در امتیازدهی دوبخشی 
)
گزارش کرده‌است.
کار دیگری
%EIS-Net: Segmenting early infarct and scoring ASPECTS simultaneously on non-contrast CT of patients with acute ischemic stroke
که ناحیه‌بندی را به کمک انطباق تصاویر انجام داده است،
برای امتیازدهی ده‌گانه و دو‌بخشی به ترتیب دقت
۸۴٪ 
و 
۹۰٪ 
را گزارش کرده‌است.
همچنین یک نمونه از قدیمی‌ترین کار‌های پیشین که دو روش برای محاسبه‌ی ASPECTS پیشنهاد داده، در روش ناحیه‌بندی و طبقه‌بندی خود،  
دقت
۹.۷۰٪
را اعلام کرده است.
%Evaluating a Deep-Learning System for Automatically Calculating the Stroke ASPECT Score
\\

چند نمونه کار پیشین نیز در ادامه عنوان می‌شود که در استخراج ویژگی‌های نواحی، از روش‌های آماری استفاده کرده‌اند.
یکی از موفق‌ترین نمونه‌ها در این دسته، پژوهشی نسبتا قدیمی است که
تشخیص
۸.۹۱٪
،
حساسیت
۲.۶۶٪
و دقت 
۹.۸۴٪
در امتیازدهی ده‌گانه و 
تشخیص
۸.٪
،
حساسیت
۸.۹۷٪
و دقت 
۹۶٪
را
در امتیازدهی دوبخشی 
به دست آورده‌است.
%Automated ASPECTS on Noncontrast CT Scans in Patients with Acute Ischemic Stroke Using Machine Learning
نمونه‌های دیگری نیز از سال‌های اخیر وجود دارند.
%Deep Learning-Based ASPECTS Scoring Method for Acute Ischemic Stroke
%An automated ASPECTS method with atlas-based segmentation
که به علت مقایسه‌پذیر نبودن و یا نامناسب بودن، از ذکر نتایج آن‌ها صرف نظر می‌شود.

\قسمت{روش ناحیه‌بندی و هم‌پوشانی}
در این روش، دو نوع ناحیه‌بندی انجام می‌شود.
نوع اول، نواحی ASPECTS و نوع دوم،
بخش‌های آسیب‌دیده‌ی مغزی در اثر انسداد عروق 
را مشخص می‌کند.
سپس هم‌پوشانی بخش‌های آسیب‌دیده با هر ناحیه محاسبه می‌شود.
در صورتی که نسبت مساحت آسیب‌دیده‌ی یک ناحیه، از یک حد آستانه فراتر برود، آن ناحیه به عنوان آسیب‌دیده گزارش می‌شود و در غیر این صورت، سالم شناخته می‌شود.
در واقع در این روش‌ها، مدل‌های یادگیری ماشین، وظیفه‌ی اصلی ناحیه‌بندی را بر عهده دارند و نه طبقه‌بندی.\\

واضح است که ناحیه‌بندی نواحی آسیب‌دیده، بر خلاف ناحیه‌بندی نواحی ده‌گانه‌ی ،ASPECTS
به روش انطباق تصاویر ممکن نیست.
زیرا الگوی ثابت و مشخصی برای نواحی آسیب‌دیده وجود ندارد.
به همین دلیل این روش‌ها برای آموزش مدل ماشین، عموما نیازمند تعداد زیادی تصویر مغزی به همراه برچسب پیکسل‌های آسیب‌دیده هستند.
این نوع از برچسب‌ها، وقت و انرژی زیادی از نیرو‌های انسانی می‌گیرند و تهیه‌ی آن‌ها دشوارتر است.\\

در میان کارهای پیشین،
سه پژوهش با
روش ناحیه‌بندی و هم‌پوشانی
یافته‌شد.
یکی از بهترین نتایج 
گزارش داده‌شده مربوط به پژوهشی در سال ۲۰۲۱ است
%Alberta Stroke Program Early CT Score Calculation Using the Deep Learning-Based Brain Hemisphere Comparison Algorithm
که
تشخیص
۹۷٪
،
حساسیت
۸۰٪
و دقت 
۹۶٪
در امتیازدهی ده‌گانه و 
تشخیص
۹۲٪
،
حساسیت
۹۸٪
و دقت 
۹۷٪
را
در امتیازدهی دوبخشی 
گزارش کرده‌است.
اما متاسفانه به وضوح اشاره نشده‌است که این نتایج مربوط به داده‌های آموزشی هستند و یا آزمایشی.
در پژوهش دیگری
تشخیص
۱۷٪
تا
۸۳٪
و
حساسیت
۶۹٪
تا
۱۰۰٪
در امتیازدهی نواحی ده‌گانه و 
تشخیص
۴۸٪
تا
۹۳٪
و
حساسیت
۷۴٪
تا
۹۹٪
در یک امتیازدهی سه‌بخشی 
ارائه شده‌است.
%Alberta Stroke Program Early CT Score Calculation Using the Deep Learning-Based Brain Hemisphere Comparison Algorithm
آخرین مورد مطالعه‌شده نیز نتایج را در قالب 
بهبودی که در توانایی تشخیصی متخصصان ایجاد شده، ذکر کرده‌است و از آن عبور می‌شود.
%Improving the diagnosis of acute ischemic stroke on non-contrast CT using deep learning a multicenter study

\قسمت{روش کل‌نگری و طبقه‌بندی}
تعداد بسیار محدودی از پژوهش‌ها در این دسته قرار می‌گیرند
که پژوهش حاضر نیز یکی از آن‌ها است.
در این روش، تنها با در دست داشتن امتیاز ASPECT کلی بیمار، تشخیص امتیاز ASPECT تصاویر فراگرفته می‌شود.
در واقع در این روش، مدل تعداد زیادی تصویر مغزی به همراه برچسب امتیاز نهایی ASPECT آن‌ها را مشاهده می‌کند و 
نمونه‌های جدید تصاویر مغزی را در یکی از دسته‌های امتیاز
۰، ۱، \dots، ۱۰
طبقه‌بندی می‌کند.\\

در فصل بعد خواهد آمد که این دسته از روش‌ها کمترین نیاز‌مندی داده‌ای را دارند.
به همین نسبت، دقت این روش‌ها نسبت به روش‌های قبلی، عموما پایین‌تر است.
با این حال، پژوهشی
%Automated ASPECTS Classification in Acute Ischemic Stroke using EfficientNetV2
وجود دارد که در این دسته از روش‌ها، بالاترین دقت 
در میان تمام کارهای پیشین
را گزارش کرده‌است.
طبق بررسی‌های انجام شده، نتایج گزارش‌شده معتبر نیستند
چرا که در ارزیابی مدل، جدایی میان داده‌های آموزشی و آزمایشی رعایت نشده‌است.
به عبارتی ارزیابی شامل داده‌هایی می‌شود که مدل، قبلا پاسخ آن‌ها را مشاهده کرده‌است و طبعا پیش‌بینی درست‌تری روی آن خواهد داشت.
لذا از ذکر و مقایسه‌ی نتایج این پژوهش صرف نظر می‌شود.\\

به این ترتیب تنها یک نمونه کار دیگر با این روش در ادبیات موضوع باقی می‌ماند.
%Evaluating a Deep-Learning System for Automatically Calculating the Stroke ASPECT Score
این مدل، امتیاز ASPECT را برای دو برش اصلی مغز می‌آموزد.
این پژوهش، خطای متوسط ۱۱۱۶.۰ و خطای واریانس 5080.2 را گزارش کرده‌است.

\قسمت{سایر روش‌ها}
طبیعتا روش‌های محاسبه‌ی خودکار ASPECTS محدود به روش‌های پیشنهادی فوق نیست
و هر پژوهشی را نمی‌توان لزوما در یکی از این دسته‌ها قرار داد.
در میان کارهای پیشین نیز چنین موردی وجود دارد.
%Deep Learning-Based Automatic Detection of ASPECTS in Acute Ischemic Stroke: Improving Stroke Assessment on CT Scans
این پژوهش که جزء کار‌های تازه‌تر است، روش جالبی را به کار برده‌است که ذکر آن خالی از لطف نیست.\\

در روش پیشنهادی این پژوهش، نواحی ASPECTS ناحیه‌بندی می‌شوند.
سپس از دو ترفند پیش‌اموزش‌دادن 
\footnote{Pretraining}
و 
تنظیم دقیق 
\footnote{Fine tuning}
برای آموزش مدل در دو مرحله استفاده کرده‌است.
در گام اول، مدل با دریافت تعداد زیادی برش از مغز به همراه برچسب ASPECTS همان برش، پیش‌آموزش می‌بیند.
سپس این مدل، با دریافت تعداد زیادی تصویر با برچسب‌هایی در سطح هر ناحیه، تنظیم دقیق می‌شود.\\

این پژوهش نهایتا تشخیص
۶.۸۱٪
،
حساسیت
۲.۶۵٪
و دقت 
۷.۷۹٪
در امتیازدهی ده‌گانه و 
تشخیص
۷.۹۰٪
،
حساسیت
۲.۷۲٪
و دقت 
۹.۸۸٪
را
در امتیازدهی دوبخشی 
گزارش می‌کند که می‌تواند در برخی کاربرد‌ها مناسب باشد.
