
\فصل{مقدمه}

در این فصل مسئله‌ی اصلی پژوهش، یعنی امتیاز ،ASPECTS به طور دقیق‌تر مورد بررسی قرار می‌گیرد و علت اهمیت بالای آن در زمینه‌ی سکته‌ی مغزی عنوان می‌شود.
سپس بررسی می‌شود که آیا چالش‌های این مسئله در پژوهش‌ها و محصولات مرتبط، حل شده‌است یا خیر و اینکه پژوهش حاضر چه مزیتی در این حوزه به همراه خواهد داشت.
در پایان نیز ساختار کلی پایان‌نامه شرح داده می‌شود.

\قسمت{تعریف مسئله}

سکته‌ی مغزی یکی از علل مهم مرگ ومیر و ناتوانی‌های اکتسابی در جهان است \cite{donkor2018stroke}.
امروزه روش‌های درمانی مختلفی برای بیماران مبتلا به این عارضه وجود دارد.
اما تجویز روش درمانی مناسب برای هر بیمار با توجه به وضعیت وی، متفاوت است.
درواقع لازم است که متخصصان، ملاک و معیاری از وضعیت پیشرفت و وخامت سکته داشته باشند تا بتوانند یک روش درمانی را برای بیمار، مناسب یا نامناسب قلمداد کنند.
یکی از مهم‌ترین این معیارها، امتیاز ASPECTS می‌باشد.

ASPECTS\footnote{The Alberta Stroke Program Early CT Score}
\cite{barber2000validity}،
یک امتیاز از ۰ تا ۱۰ است که معیاری از وخامت سکته در بیماران را به دست می‌دهد.
درواقع این امتیاز از بررسی وضعیت ۱۰ ناحیه‌ی مغزی، که در دو نیم‌کره‌ی مغز به صورت متقارن وجود دارند، محاسبه می‌شود.
در صورتی که هیچ عارضه‌ی ایسکمیک\footnote{Ischemic}
 در مغز وجود نداشته‌باشد، امتیاز ASPECTS برابر ۱۰ خواهد بود و به ازای هر ناحیه‌ی آسیب‌دیده، یک امتیاز از ۱۰ کم می‌شود.
به این ترتیب بیماری که هر ۱۰ ناحیه‌ی ASPECTS او، حداقل در یک نیم‌کره، آسیب دیده‌باشد، امتیاز صفر را دریافت خواهد کرد.
این امتیاز سپس می‌تواند معیاری در اختیار متخصصان قرار دهد که تشخیص بدهند آیا درمان 
لخته‌زدایی مکانیکی\footnote{\lr{Mechanical Thrombectomy}}
برای بیمار مناسب است یا خیر.

نکته‌ی حائز اهمیت آن است که امتیازدهی ،ASPECTS حتی برای متخصصین این حوزه، یک امر چالش‌بر‌انگیز است.
به نحوی که در یک مطالعه \cite{van2021aspects}،
میزان توافق میان امتیازدهندگان، تنها ۲۸٪ محاسبه شده‌است.
از طرفی، نشان داده‌شده‌است که ابزار‌های محاسبه‌ی خودکار ،ASPECTS می‌توانند
میزان این توافق و سرعت امتیازدهی متخصصان را افزایش دهند \cite{chen2022improving}.
به همین جهت، این پژوهش قصد دارد با ارائه‌ی یک روش خودکار تشخیص امتیاز دوبخشی ASPECTS 
در راستای این بهبود دقت و سرعت، راهگشا باشد.

\قسمت{اهمیت موضوع}
میان دقت، سرعت و دسترس‌پذیری در تشخیص سکته‌ی مغزی، یک 
بده‌بستان\footnote{Tradeoff}
وجود دارد.
یک‌سری تصاویر مانند ،MRI علائم سکته را بهتر در خود نمایان کرده و تشخیص را برای متخصصان ساده‌تر می‌کنند.
اما اخذ این تصاویر، زمان زیادی نیاز دارد و ممکن است در تمام مراکز تصویر‌برداری نیز در دسترس نباشند.
از سوی دیگر، تصاویر ،CT علائم سکته را کمتر مشخص می‌کنند و باعث می‌شوند که تشخیص، سخت‌تر و توافق میان تشخیص‌دهندگان کمتر شود.
اما مزیت این مدل تصویربرداری، در سرعت اخذ تصویر و کاربرد فراگیر آن در اکثر مراکز تصویر برداری است.

اصطلاحی در این حوزه وجود دارد که عنوان می‌کند "زمان، مغز است" \cite{saver2006time}.
این جمله به اهمیت زمان و لزوم تشخیص و درمان سریع سکته‌ی مغزی اشاره می‌کند.
به طور متوسط، در بیمارانی که دچار سکته‌ی مغزی ایسکمیک شده‌اند، در هر دقیقه، ۹.۱ میلیون سلول عصبی از بین می‌رود.
این عدد در مقایسه با نرخ عادی از بین رفتن سلول‌های عصبی، مانند آن است که مغز در یک ساعت، به مدت ۶.۳ سال عمر کرده‌است \cite{saver2006time}.
به همین جهت، سرعت عمل در تشخیص سکته‌ی مغزی و آغاز هر چه زودتر درمان آن، امری حیاتی است.
در نتیجه در بده‌بستان میان دقت و سرعت، این سرعت است که برتری می‌یابد و تصویربرداریسی‌تی‌اسکن و روش‌های تشخیصی ممکن بر روی آن را غالب می‌کند.

امتیازدهی ASPECTS نیز یک روش تشخیصی مبتنی بر تصاویرسی‌تی‌اسکن است.
به همین دلیل است که پژوهش حول این مسئله، از اهمیت بالایی برخوردار است.
اما همانطور که پیش‌تر ذکر شد، علی‌رغم سرعت بالای تشخیص در این روش، افزایش دقت حاصل از آن، یک موضوع چالش‌برانگیز است.
عدم توافق بالا میان تشخیص متخصصان نیز خبر از این مشکل دارد.
مشکلی که همچنان میان متخصصان انسانی وجود دارد.
هوش مصنوعی و روش‌های یادگیری ماشین می‌توانند به حل این مشکل کمک کنند.
پژوهش‌هایی انجام شده‌است
که نشان می‌دهد تشخیص خودکار امتیاز ASPECTS می‌تواند توافق میان متخصصان را افزایش بدهد \cite{chen2022improving}.
بنابراین ضروری است که این روش‌ها، با افزایش هرچه بیشتر دقت، در راستای بهبود سرعت و دقت خدمات درمانی سکته‌ی مغزی، کمک‌کننده باشند.

\قسمت{ادبیات موضوع}

فعالیت‌هایی که به طور مستقیم در حوزه‌ی امتیازدهی ASPECTS انجام می‌شوند را می‌توان در دو دسته‌ی کلی بررسی کرد.
دسته‌ی اول، 
برنامه‌های کاربردی\footnote{Application} هستند 
که به صورت تجاری عرضه شده و در حال استفاده در مراکز درمانی می‌باشند.
از جمله‌ی این برنامه‌ها می‌توان به RapidAI \cite{rapidai}،
 Viz.ai \cite{vizai} و e-ASPECTS \cite{brainomix} اشاره کرد.
بعضاً این برنامه‌ها بر روی چندین میلیون تصویر از بیش از ۱۰۰ کشور دنیا آموزش دیده‌اند
و به دقت بسیار مطلوبی دست یافته‌اند \cite{rapidai}.

دسته‌ی دوم شامل پژوهش‌هایی می‌شود که بر روی
تعداد تصویرهای بسیار کوچک‌تری کار می‌کنند.
مجموعه‌داده‌هایی\footnote{Dataset}
 که محدود به یک یا چند مرکز درمانی می‌شوند و از نظر تنوع و تعداد،‌ با برنامه‌های فوق‌الذکر قابل مقایسه نیستند.
این پژوهش‌ها سعی دارند روش‌های جدید برای تشخیص ASPECTS ارائه دهند و یا توانایی مدل‌های یادگیری پیشین را بر روی مسئله‌ی ASPECTS بررسی کنند.
 این مطالعات و پژوهش‌های انجام‌شده، هر یک با در نظر 
 گرفتن محدودیت‌های  موجود، مورد ارزیابی قرار 
 می‌گیرند.
روش‌های پیش‌رو، سپس می‌توانند در هسته‌ی محاسباتی برنامه‌های تجاری قرار بگیرند 
و با استفاده از ظرفیت‌های داده‌ای و محاسباتی موجود، نتایج بهتری را ارائه دهند.

بنابراین، دو دسته فعالیتی که در حوزه‌ی ASPECTS معرفی شد، یعنی برنامه‌های کاربردی توانمند و فعالیت‌های پژوهشی، هر دو مورد نیاز هستند و به نحوی مکمل هم می‌باشند.
پژوهش حاضر، در دسته‌ی دوم این فعالیت‌ها قرار می‌گیرد
و در ادامه‌ی این نوشتار نیز، تنها پژوهش‌های مطالعاتی انجام شده در حوزه‌ی ASPECTS مورد بررسی، ارجاع و مقایسه قرار خواهند گرفت.

در فصل سوم، این پژوهش‌ها به تفصیل بیشتری مورد 
بحث قرار می‌‌گیرند و
محدودیت‌ها و مزیت‌های هر یک بررسی می‌شود.
به طور کلی، کار‌های پیشین از نظر میزان داده‌ی موجود، نوع اطلاعات 
برچسب\footnote{Label}
داده‌ها و اطلاعات خروجی،
قابل دسته‌بندی و مقایسه هستند.
در این فصل نشان داده‌خواهدشد که پژوهش حاضر، یکی از معدود مطالعاتی است که با محدودیت‌های داده‌ای مشابه انجام شده‌است و 
در این زمینه به نتایج بسیار مطلوبی دست یافته‌است.

\قسمت{معیار‌های ارزیابی}
با توجه به نتایج گزارش‌شده در کارهای پیشین،
مدل پیشنهادی در این پروژه بر اساس تعدادی از معیار‌های پرکاربرد یادگیری ماشین  
در حوزه‌ی طبقه‌بندی تصاویر 
ارزیابی می‌شود.
این معیارها عبارتند از
 دقت،\footnote{Accuracy}
حساسیت،\footnote{Sensitivity}
تشخیص،\footnote{Specificity}
صحت،\footnote{Precision}
بازیابی\footnote{Recall}
و 
مساحت زیر نمودار مشخصه‌ی عملیاتی گیرنده\footnote{AUC}.
اگرچه هرکدام از این معیار‌ها اطلاعات به‌خصوصی را در رابطه با ویژگی‌های مدل به دست می‌دهند، دو معیار دقت و AUC می‌توانند یک درک کلی از توانایی تشخیصی مدل را فراهم کنند.
در فصل آزمایش‌ها، عملکرد مدل پیشنهادی در هر یک از این معیار‌ها گزارش خواهد شد.

\قسمت{اهداف پژوهش}

پیش‌تر ذکر شد که محدودیت‌های داده‌ای، تاثیر به‌سزایی در توانایی و عملکرد روش‌های یادگیری ماشین دارند.
یکی از مهم‌ترین چالش‌های حوزه‌ی یادگیری ماشین نیز در کسب بهترین نتایج از داده‌های محدود - چه از نظر کمی و چه از نظر کیفی - می‌باشد.
از طرفی فراهم کردن مجموعه‌داده‌های بزرگی که توسط متخصصان به صورت جزئی برچسب‌گذاری شده‌باشند، امری دشوار، زمان‌بر و گاه غیر عملی است.
بنابراین، ارائه‌ی روش‌هایی که بتوانند از حداکثر قابلیت‌های چنین مجموعه‌داده‌هایی استفاده کنند از اهمیت بالایی برخوردار است.

این پژوهش در وهله‌ی اول می‌کوشد تا ظرفیت موجود در داده‌های یکی از مراکز درمانی را در زمینه‌ی تشخیص سکته‌ی مغزی بسنجد و سپس روش کارآمد و تکرارپذیری را در حوزه‌ی یادگیری تصاویر پزشکی، به طور خاص محاسبه‌ی ،ASPECTS ارائه کند.
این روش علی‌رغم محدودیت‌های موجود، به عملکرد قابل مقایسه‌ای با کارهای مشابه دست یافته‌است و
به علت جامعیت بالا، با تنظیمات جزئی، قابل اعمال بر روی سایر کاربرد‌های پزشکی می‌باشد.


\قسمت{ساختار پایان‌نامه}

این پایان‌نامه در شش فصل به شرح زیر ارائه می‌شود.
برخی مفاهیم اولیه در رابطه با سکته‌ی مغزی ایسکمیک و امتیاز ASPECTS در فصل دوم به اختصار ارائه خواهد‌شد.
این مفاهیم از آن جهت اهمیت دارند که انطباق ساختار مدل ارائه شده با روش های مورد استفاده‌ی متخصصان را بهتر مشخص می‌کند.
همچنین در درک روش‌های مختلف ارائه‌شده در کارهای پیشین و نیامندی‌های داده‌ای هر یک راهگشا خواهد بود.

فصل سوم به مطالعه و بررسی کارهای پیشین مرتبط با امتیاز‌دهی خودکار ASPECTS می‌پردازد.
در فصل چهارم، روش پیشنهادی در پژوهش حاضر شرح داده می‌شود 
و در بخش پنجم، نتایج حاصله از این روش عنوان می‌شوند.
در نهایت،‌ فصل ششم به جمع‌بندی کارهای انجام شده، موفقیت‌ها و ناکارآمدی‌های متصور برای این پژوهش و ارائه‌ی پیشنهادهایی برای انجام کارهای آتی خواهد پرداخت.