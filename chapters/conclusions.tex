
\فصل{نتیجه‌گیری}

در این فصل با جمع‌بندی نتایج روش پیشنهادی، نقاط قوت و کاستی‌های آن، جایگاه این پژوهش در حوزه‌ی تشخیص ASPECTS 
مورد بحث قرار می‌گیرد.
در انتها نیز پیشنهاداتی برای پژوهش‌های مرتبط با این حوزه و یا ادامه‌ی پروژه‌ی حاضر ارائه می‌گردد. 


\قسمت{نتایج اعتبارسنجی مدل}

نتایجی که در فصل گذشته ارائه شد، نشان می‌دهد که این پروژه از دیدگاه پژوهشی، جزء مطالعات پیش‌رو محسوب می‌شود.
در فصل کارهای پیشین ذکر شد که در جستجوی تحقیقات حوزه‌ی ،ASPECTS
پژوهش معتبری با نوع برچسب و محدودیت مشابه این پروژه به دست نیامده‌است.
بنابراین مقایسه‌ی نتایج این پروژه با پژوهش‌های دیگر به اندازه‌ی کافی دقیق نخواهد بود.
با این حال مشاهده می شود که
عملکرد مدل پیشنهادی، علی‌رغم تمام محدودیت‌های موجود، در کلاس پژوهش‌هایی با مجموعه‌داده‌های بزرگ‌تر قرار می‌گیرد.\\

از جمله‌ی این پژوهش‌ها  
می‌توان به 
\cite{cao2022deep} اشاره کرد که 
پژوهش متن‌بازی بر روی بیش از هزار بیمار در سال ۲۰۲۲ بوده و دقت دوبخشی و AUC در حدود 91.0 گزارش کرده‌است.
نمونه‌ی دیگر، پژوهش تازه‌تری در سال ۲۰۲۳ با تصاویر بیش از ۳۰۰ بیمار می‌باشد که AUC امتیازدهی دوبخشی ($leq 4$) را 89.0 اعلام کرده‌است \cite{lee2023clinical}.
علاوه بر این موارد، دو پژوهش دوبخشی دیگر با اطلاعات بیش از ۲۵۰ بیمار، در سال‌های ۲۰۲۲ و ۲۰۲۱، به ترتیب دقت
889.0 و
90.0
و
AUC
82.0 
و 
87.0 
را اعلام کرده‌اند \cite{chiang2022deep,kuang2021eis}.
یک پژوهش نیز با محدودیت داده‌ای تقریبا مشابه، دقتی در حدود 70.0 گزارش نموده‌است \cite{yu2021automated}.
در این بین، یک پژوهش قدیمی‌تر، مربوط به چند سال قبل وجود دارد که دقت 96.0 و AUC 89.0 را در بخش نتایج دو‌بخشی ($\leq 4$) ارائه کرده‌است \cite{kuang2019automated}.\\

البته همانطور که ذکر شد، نتایج بالای روش پیشنهادی تنها در حوزه‌ی پژوهشی پیش‌رو هستند و بدیهی است که در حوزه‌ی کاربرد پزشکی
به حد اطمینان قابل قبولی دست نیافته‌اند.
چرا که در فاز عملیاتی، کوچک‌ترین تشخیص نادرستی می‌تواند آسیب‌های جبران ناپذیری به افراد وارد کند.
بعلاوه در این پژوهش، عملکرد مدل بر روی داده‌های موجود بهینه‌سازی شده‌است و دقت‌های گزارش شده، بالاترین دقت‌های ممکن بر روی این مجموعه‌داده هستند.
این بدان معناست که مدل در مواجهه با داده‌های جدید ممکن است رفتار متفاوتی از خود نشان دهد و کوچک بودن نمونه‌ی در دسترس آن، این احتمال را بیش‌تر هم می‌کند.
در نتیجه ضروری است که مدل بر روی تعداد بیش‌تری داده آموزش و آزمایش شود.
در این صورت ممکن است مدل خروجی بتواند به عنوان یک دستیار در تشخیص انسانی مورد استفاده قرار بگیرد.
که البته این خود نیاز به بررسی‌های بیش‌تر دارد و موضوع تحقیق برخی مقالات در این زمینه بوده‌است.\\

لازم است توجه شود که قابل اطمینان نبودن یک مدل برای کاربرد مستقیم پزشکی، تنها به نحوه‌ی طراحی آن مدل مربوط نمی‌شود.
بلکه گاه از ذات مسئله و امکانات موجود برمی‌آید.
پیش‌تر ذکر شد که دانش مدل در برچسب‌های سطح مغز، مبتنی بر مشاهده‌ی حالت‌های مختلف بروز سکته می‌باشد.
اما در حجم موجود از تصاویر، مشخصا تمام حالت‌ها جای نگرفته‌اند.
بناربراین حتی اگر دقت گزارش شده از یک مدل به حد بالایی رضایت‌بخش باشد، باید به محدود بودن دانش آن و 
تعمیم‌ناپذیری آن در داده‌های جدید نیز توجه داشت.

\قسمت{کاستی‌ها}
همانطور که پیش‌تر ذکر شد، مهم‌ترین کاستی این پژوهش، حجم محدود مجموعه‌داده و برچسب‌های سطح بالای (برچسب تک‌عددی صفر تا ۱۰ در سطح مغز) آن بوده‌است.
چالش دیگری که در رابطه با مجموعه‌داده وجود داشته، نامتوازن بودن امتیازات ASPECT آن بوده‌است.
به نحوی که در کل داده‌ها، تنها ۲۵ بیمار با امتیازات صفر تا ۵ موجود بوده‌است.
این موضوع ممکن است بر نتایج به دست آمده از مدل تاثیر داشته باشد و نیاز است مورد آزمایش بیش‌تری قرار بگیرد.
همچنین چنانکه در قسمت قبل آمد، تعداد محدود داده‌ها موجب نبود حالت‌های بسیاری از بروز سکته در تصاویر عرضه‌شده به مدل شده‌است.
بنابراین لازم است با افزایش تعداد و تنوع تصاویر ورودی، 
علاوه بر افزایش دانش مدل، عملکرد آن نیز مورد ارزیابی دوباره قرار بگیرد.\\

نکته‌ی دیگری که لازم است مورد توجه قرار بگیرد، حذف داده‌های دارای نویز شدید و یا کیفیت پایین بوده‌است.
درواقع مجموعه‌داده‌‌ی موجود، حاصل از هرس داده‌های اولیه و جدا کردن تصاویر با وضوح و مشخصات پایین می‌باشد.
این موضوع ممکن است در کاربرد عملی که تصاویر با نویز بالا و کیفیت پایین وجود دارند، عملکرد مدل بسیار متفاوت باشد.
علاوه بر این موضوع، داده‌های آموزشی و آزمایشی
این پژوهش، هر دو از یک مجموعه هستند و شباهت زیادی با هم دارند.
عملکرد مدل پیشنهادی این پروژه در مواجهه با داده‌هایی از مراکز تصویر برداری متنوع‌تر ارزیابی نشده‌است.\\

نکته‌ی پایانی در این بخش، در رابطه با زمان اخذ تصاویر می‌باشد.
معیاری در پژوهش‌های حوزه‌ی سکته‌ی مغزی وجود دارد که زمان از لحظه‌ی بروز علائم تا تصویر برداری را مشخص می‌کند.
مجموعه‌داده‌ی پژوهش حاضر از این جهت مورد تفکیک قرار نگرفته‌است و لازم است اطلاعات دقیق‌تری از این معیار کسب و در آموزش و آزمایش مدل، یک‌دست‌سازی شود.\\



