
\فصل{نتیجه‌گیری}

در این فصل با جمع‌بندی نتایج روش پیشنهادی، نقاط قوت و کاستی‌های آن، جایگاه این پژوهش در حوزه‌ی تشخیص ASPECTS 
مورد بحث قرار می‌گیرد.
در انتها نیز پیشنهاداتی برای پژوهش‌های مرتبط با این حوزه و یا ادامه‌ی پروژه‌ی حاضر ارائه می‌گردد. 


\قسمت{کاربردپذیری}

نتایجی که در فصل گذشته ارائه شد، نشان می‌دهد که این پژوهش در میان کارهای پیشین، جزء مطالعات پیشرو محسوب می‌شود
البته همانطور که ذکر شد، نتایج برجسته‌ی روش پیشنهادی این پروژه، مربوط به حوزه‌ی پژوهشی می‌باشد و در حوزه‌ی کاربرد پزشکی
به حد اطمینان قابل قبولی نرسیده‌اند.
چرا که در فاز عملیاتی، کوچک‌ترین تشخیص نادرستی می‌تواند آسیب‌های جبران ناپذیری به افراد وارد کند.
بعلاوه در این پژوهش، عملکرد مدل بر روی داده‌های موجود بهینه‌سازی شده‌است و دقت‌های گزارش شده، بالاترین دقت‌های ممکن بر روی این مجموعه‌داده هستند.
این بدان معناست که مدل در مواجهه با داده‌های جدید ممکن است رفتار متفاوتی از خود نشان دهد و کوچک بودن نمونه‌ی در دسترس آن، این احتمال را بیشتر هم می‌کند.

در نتیجه ضروری است که مدل بر روی تعداد بیشتری داده آموزش و آزمایش شود.
در این صورت ممکن است مدل خروجی بتواند به عنوان یک دستیار در تشخیص انسانی مورد استفاده قرار بگیرد.
که البته این خود نیاز به بررسی‌های بیشتر دارد و موضوع تحقیق برخی مقالات در این زمینه بوده‌است.

لازم است توجه شود که قابل اطمینان نبودن یک مدل برای کاربرد مستقیم پزشکی، تنها به نحوه‌ی طراحی آن مدل مربوط نمی‌شود.
بلکه گاه از ذات مسئله و امکانات موجود برمی‌آید.
پیش‌تر ذکر شد که دانش مدل در برچسب‌های سطح مغز، مبتنی بر مشاهده‌ی حالت‌های مختلف بروز سکته می‌باشد.
اما در حجم موجود از تصاویر، مشخصاً تمام حالت‌ها جای نگرفته‌اند.
بناربراین حتی اگر دقت گزارش شده از یک مدل به حد بالایی رضایت‌بخش باشد، باید به محدود بودن دانش آن و 
تعمیم‌ناپذیری آن در داده‌های جدید نیز توجه داشت.

\قسمت{نقاط قوت}

اولین نکته‌ی مثبت در رابطه با مطالعه‌ی انجام‌شده، 
جامعیت و قابلیت بازاستفاده‌ی روش پیشنهادی آن در حوزه‌ی پردازش تصاویر پزشکی است.
مراحل پیش‌پردازشی پیشنهادی این پروژه، به‌ویژه روش پویای ارائه‌شده برای افزایش وضوح تصاویر، می‌تواند عیناً یا با تغییر اندک در پژوهش‌های دیگر بر روی تصاویر پزشکی، به خصوص تصاویر مغزی،
مورد استفاده قرار بگیرند. 

علاوه بر این، 
ساختار مدل پیشنهادی نیز خاص امتیازدهی ASPECTS نبوده و قابل اعمال برای سایر تشخیص‌های پزشکی بر روی تصاویر می‌باشد.
درواقع پژوهش حاضر توانسته نشان بدهد که ویژگی‌های استخراج‌شده توسط مدل پیش‌آموزش‌دیده‌ی \lr{EfficientNetB0} برای مقاصد طبقه‌بندی تصاویر مغزی مناسب هستند.
همچنین طی یک جستجوی نظام‌مند، توانسته یک ساختار شبکه‌ی طبقه‌بندی‌کننده‌ی مناسب بر روی این ویژگی‌ها ارائه کند که می‌تواند یک الگوی اولیه برای شبکه‌ی پژوهش‌هایی باشد که در این زمینه از یادگیری انتقالی استفاده می‌کنند.

نقطه‌ی قوت دیگر، انطباق روش پیشنهادی این پروژه با روش انسانی تشخیص ASPECTS می‌باشد.
طی جستجو‌هایی که در کارهای پیشین به انجام رسیده‌است،
این پژوهش، اولین موردی است که تشخیص دو‌بعدی ASPECTS را به جای دو برش، بر روی شش برش اصلی حاوی نواحی ASPECTS انجام می‌دهد.
این ساختار، پایه‌ی دانشی مدل را افزایش داده و نتایج آن را قابل اطمینان‌تر می‌کند.

همچنین این پروژه توانسته‌است روش نسبتاً نوینی را برای 
آموزش و آزمایش مدل‌های طبقه‌بندی دوبخشی با مجموعه‌داده‌‌های نامتوازن ارائه دهد.
 این پروژه با یک مسئله‌ی طبقه‌بندی دو-کلاسه مواجه بوده است.
 در حالی که مجموعه‌داده‌ی موجود، اطلاعاتی بیش از ASPECTS دوبخشی را شامل می‌شد
 که می‌توانست مدل را در یادگیری بهتر الگوها یاری کند.
 اما تعداد داده‌های اندک برای برخی امتیازات در این مجموعه‌داده، مانع از آموزش و آزمایش کامل ۱۱-کلاسه می‌شد.
 این پژوهش با 
 طراحی حالت میانه‌ای\footnote{تمام امتیازات بخش $<6$ در قالب یک کلاس گردآوری شده و آموزش بر روی این کلاس تجمیعی و ۵ کلاس امتیازات $\geq 6$ انجام شد.}
 از آموزش دو‌کلاسه و چند-کلاسه 
 نشان داد که استفاده از این اطلاعات افزوده می‌تواند 
 یادگیری و نتایج مدل در فاز آزمایشی را بهبود ببخشد.

\قسمت{کاستی‌ها}
همانطور که پیش‌تر ذکر شد، مهم‌ترین کاستی این پژوهش، حجم محدود مجموعه‌داده و برچسب‌های سطح بالای (برچسب تک‌عددی صفر تا ۱۰ در سطح مغز) آن بوده‌است.
چالش دیگری که در رابطه با مجموعه‌داده وجود داشته، نامتوازن بودن امتیازات ASPECTS آن بوده‌است.
به نحوی که در کل داده‌ها، تنها ۲۵ بیمار با امتیازات صفر تا ۵ موجود بوده‌است.
این موضوع ممکن است بر نتایج به‌دست آمده از مدل تاثیر داشته باشد و نیاز است مورد آزمایش بیشتری قرار بگیرد.

همچنین چنانکه در قسمت قبل آمد، تعداد محدود داده‌ها موجب شده حالت‌های بسیاری از بروز سکته در تصاویر عرضه‌شده به مدل وجود نداشته باشند.
بنابراین لازم است با افزایش تعداد و تنوع تصاویر ورودی، 
علاوه بر افزایش دانش مدل، عملکرد آن نیز مورد ارزیابی دوباره قرار بگیرد.

نکته‌ی دیگری که لازم است مورد توجه قرار بگیرد، حذف داده‌های دارای نویز شدید و یا کیفیت پایین بوده‌است.
درواقع مجموعه‌داده‌‌ی موجود، حاصل از هرس داده‌های اولیه و جدا کردن تصاویر با وضوح و مشخصاًت پایین می‌باشد.
در نتیجه ممکن است در کاربرد عملی که تصاویر با نویز بالا و کیفیت پایین وجود دارند، عملکرد مدل بسیار متفاوت باشد.

علاوه بر این موضوع، داده‌های آموزشی و آزمایشی
این پژوهش، هر دو از یک مجموعه هستند و شباهت زیادی با هم دارند.
با توجه به اینکه در این پروژه از روش ارزیابی متقابل برای آموزش و سنجش مدل استفاده شده‌است، این امکان وجود دارد که قابلیت‌های مدل تنها بر روی همین مجموعه‌داده بهینه شده‌باشند.
بنابراین لازم است که
عملکرد مدل پیشنهادی این پروژه در مواجهه با داده‌هایی دیده‌نشده از مراکز تصویر برداری متنوع‌تر ارزیابی گردد.

نکته‌ی پایانی در این بخش، در رابطه با زمان اخذ تصاویر می‌باشد.
معیاری در پژوهش‌های حوزه‌ی سکته‌ی مغزی وجود دارد که زمان از لحظه‌ی بروز علائم تا تصویر برداری را مشخص می‌کند.
مجموعه‌داده‌ی پژوهش حاضر از این جهت مورد تفکیک قرار نگرفته‌است و لازم است اطلاعات دقیق‌تری از این معیار کسب و در آموزش و آزمایش مدل، یک‌دست‌سازی شود.

\قسمت{پیشنهاد ادامه‌ی کار}

با در نظر گرفتن کاستی‌های عنوان‌شده، می‌توان چند پژوهش تکمیلی در حوزه‌ی تشخیص ASPECTS را در ادامه‌ی این پژوهش متصور شد.
پژوهش حاضر توانسته با امکانات داده‌ای موجود، عملکرد قابل مقایسه‌ای با پژوهش‌های موجود ارائه دهد.
از طرفی
آموزش مدل‌های تشخیص ASPECTS بر روی برچسب‌های سطح ناحیه و سطح پیکسل، با توجه به دانش بالاتری که به مدل می‌دهد، می‌تواند عملکرد اطمینان‌بخش‌تری را ارائه دهد.
بنابراین بازسازی و آزمایش روش پیشنهادی بر روی برچسب‌های سطوح بالاتر می‌تواند در کسب نتایج بهتر و مطمئن‌تر راهگشا باشد.
حتی آموزش مدل موجود بر روی تعداد و تنوع بالاتری از مجموعه‌داده ممکن است نتایج جدیدی را آشکار کند.

همچنین یک مورد از مطالعاتی که حول پروژه‌ی حاضر می‌تواند انجام شود، ارزیابی تاثیر استفاده از این مدل بر تشخیص ASPECTS متخصصان و 
میزان توافق میان آنان در تشخیص است.
چنانکه پیش از این نیز مطالعات مشابهی بر روی روش‌های ارائه‌شده‌ی ASPECTS انجام شده‌است.
انجام چنین تحقیقاتی برای هدایت روش پیشنهادی به سمت کاربرد‌های عملی، ضروری است.

پژوهش دیگری که می‌تواند مسیر را برای مطالعات آینده در حوزه‌ی ASPECTS هموار کند، 
طبقه‌بندی خودکار برش‌های مغزی بر اساس نواحی ASPECTS است.
به عبارتی استفاده از یادگیری ماشین برای جداسازی تصاویر مغزی از غیر مغزی و جداسازی ۶ برش ویژه‌ی مورد بررسی از میان تصاویر مغزی.
در پژوهش حاضر، به علت حجم محدود داد‌ه‌ها، این طبقه‌بندی به صورت انسانی انجام شده‌است.
اما چنین ابزاری علاوه بر 
کمک به ایجاد مجموعه‌داده‌های بزرگ در پژوهش‌های آتی، در عملیاتی‌کردن پروژه‌ی حاضر نیز مفید خواهد بود.
مجموعه‌داده‌ی مورد استفاده در این پژوهش، خود می‌تواند پایه‌ای بر یادگیری مدل مذکور باشد.
