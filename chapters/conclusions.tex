
\فصل{نتیجه‌گیری}

در این فصل با جمع‌بندی نتایج روش پیشنهادی، نقاط قوت و کاستی‌های آن، جایگاه این پژوهش در حوزه‌ی تشخیص ASPECTS 
مورد بحث قرار می‌گیرد.
در انتها نیز پیشنهاداتی برای پژوهش‌های مرتبط با این حوزه و یا ادامه‌ی پروژه‌ی حاضر ارائه می‌گردد. 


\قسمت{نتایج اعتبارسنجی مدل}

نتایجی که در فصل گذشته ارائه شد، نشان می‌دهد که این پروژه از دیدگاه پژوهشی، جزء مطالعات پیش‌رو محسوب می‌شود.
در فصل کارهای پیشین ذکر شد که در جستجوی تحقیقات حوزه‌ی ،ASPECTS
پژوهش معتبری با نوع برچسب و محدودیت مشابه این پروژه به دست نیامده‌است.
بنابراین مقایسه‌ی نتایج این پروژه با پژوهش‌های دیگر به اندازه‌ی کافی دقیق نخواهد بود.
با این حال مشاهده می شود که
عملکرد مدل پیشنهادی، علی‌رغم تمام محدودیت‌های موجود، در کلاس پژوهش‌هایی با مجموعه‌داده‌های بزرگ‌تر قرار می‌گیرد.\\

از جمله‌ی این پژوهش‌ها  
می‌توان به 
\cite{cao2022deep} اشاره کرد که 
پژوهش متن‌بازی بر روی بیش از هزار بیمار در سال ۲۰۲۲ بوده و دقت دوبخشی و AUC در حدود 91.0 گزارش کرده‌است.
نمونه‌ی دیگر، پژوهش تازه‌تری در سال ۲۰۲۳ با تصاویر بیش از ۳۰۰ بیمار می‌باشد که AUC امتیازدهی دوبخشی ($leq 4$) را 89.0 اعلام کرده‌است \cite{lee2023clinical}.
علاوه بر این موارد، دو پژوهش دوبخشی دیگر با اطلاعات بیش از ۲۵۰ بیمار، در سال‌های ۲۰۲۲ و ۲۰۲۱، به ترتیب دقت
889.0 و
90.0
و
AUC
82.0 
و 
87.0 
را اعلام کرده‌اند \cite{chiang2022deep,kuang2021eis}.
یک پژوهش نیز با محدودیت داده‌ای تقریبا مشابه، دقتی در حدود 70.0 گزارش نموده‌است \cite{yu2021automated}.
در این بین، یک پژوهش قدیمی‌تر، مربوط به چند سال قبل وجود دارد که دقت 96.0 و AUC 89.0 را در بخش نتایج دو‌بخشی ($\leq 4$) ارائه کرده‌است \cite{kuang2019automated}.\\





