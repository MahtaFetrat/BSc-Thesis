
\فصل{روش پیشنهادی}

فصل حاضر به شرح مراحل عملیاتی پروژه اختصاص دارد.
در این بخش ابتدا نحوه‌ی آماده‌سازی مجموعه‌داده شرح داده می‌شود.
سپس بررسی می‌شود که این تصاویر دستخوش چه تغییراتی شده و چگونه به مدل ورودی داده می‌شوند.
در ادامه، طراحی انجام‌شده برای مدل یادگیری ماشین توصیف می‌شود.
بخش انتهایی این بخش نیز به ذکر 
جزئیات فرایند آموزش مدل و مقدمه‌ای بر فرایند آزمایش مدل خواهد گذشت.

% \قسمت{روش ناحیه‌بندی و طبقه‌بندی}
% در این روش، ده بخش مورد توجه ASPECTS در تصاویر مغزی ناحیه‌بندی
% \footnote{Segmentation}
%  می‌شوند.
% به این ترتیب،  
% مدل یادگیری ماشین به طور مستقیم از محل این نواحی در تصاویر آگاهی می‌یابد.
% سپس مدل آموزش داده می‌شود که هر ناحیه‌ای که می‌بیند، آیا آسیب دیده‌است یا خیر.
% یعنی یاد می‌گیرد که هر ناحیه را به دو دسته‌ی آسیب‌دیده و سالم طبقه‌بندی
% \footnote{Classification}
% کند.
% در نهایت، ، سپس با جمع امتیازات تمام ده ناحیه‌ی هر بیمار، امتیاز ASPECTS وی به دست می‌آید.\\

% \subsection{ناحیه‌بندی نواحی ASPECTS}
% ناحیه‌بندی ۱۰ بخش ASPECTS تصاویر به دو طریق مختلف انجام می شود.
% روش اول از یادگیری ماشین بهره می‌گیرد.
% در این روش، هر تصویر مغزی، برچسبی دارد که نشان می‌دهد کدام پیکسل‌های تصویر متعلق به هر ناحیه هستند.
% تعداد زیادی از تصاویر مغزی به همراه این برچسب‌ها به مدل ورودی داده تا ناحیه‌بندی را بیاموزد.
% به این ترتیب، مدل می‌تواند با دریافت یک تصویر مغزی جدید و بدون برچسب، مشخص کند که کدام پیکسل‌ها متعلق به هر ناحیه هستند.\\
