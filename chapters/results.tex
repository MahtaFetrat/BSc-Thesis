
\فصل{نتایج جدید}

در این فصل، نتایج به‌دست‌آمده از اعتبارسنجی مدل ذکر می‌شود.
این نتایج به تفکیک هر لایه‌ی اعتبارسنجی گزارش می‌شوند و جمع‌بندی عملکرد مدل بر روی کل مجموعه‌داده نیز عنوان می‌شود.

\قسمت{روش اعتبارسنجی}

همانطور که پیش‌تر در فصل مفاهیم اولیه اشاره شد و در فصل روش پیشنهادی تشریح گردید، در این پژوهش برای ارزیابی مدل از روش 
اعتبارسنجی متقابل ۵ لایه‌ای\footnote{\lr{5-fold Cross Validation}}
استفاده می‌شود.
در این روش، مجموعه‌داده به ۵ قسمت تقسیم می‌شود.
به ازای هر قسمت، مدل یک مرتبه آموزش می‌بیند.
به عبارتی در هر مرتبه،  مدل بر روی یک‌پنجم انتخاب‌شده‌ی داده‌ها آزمایش می‌شود و از چهار‌پنجم باقی‌مانده‌ی آن‌ها می‌آموزد.
به این ترتیب پس از پایان ۵ مرتبه آموزش و آزمایش، مدل بر روی کل مجموعه‌داده‌ی در دسترس آزمایش و اعتبارسنجی شده‌است.

در فصل روش پیشنهادی اشاره شد که افراز ۵ لایه‌ای انجام‌شده بر روی مجموعه‌داده، تقریبا متوازن است.
به این معنا که از هر امتیاز ASPECT تقریبا تعداد یکسانی در هر قسمت وجود دارد.
به این ترتیب مدل در فرایند آموزش هر یک از ۵ لایه، تعداد نمونه‌ی مناسبی از هر امتیاز را مشاهده کرده و فرامی‌گیرد.
جدول \ref{جدول:test-folds} اندازه‌ی مجموعه‌ی آموزشی هر لایه و توزیع امتیازات ASPECT در آن‌ها را نشان می‌دهد.

\شروع{لوح}[ht]

\vspace{1.5em}

\تنظیم‌ازوسط
\شرح[اطلاعات مجموعه‌ی آزمایشی پنج لایه‌ی اعتبارسنجی]{حجم و ترکیب امتیازهای ASPECT مجموعه‌ی آزمایشی در پنج لایه‌ی اعتبارسنجی.}

\begin{tabular}{cccccccc}
    \cline{3-8}
                                                               &                                        & \multicolumn{6}{c}{تعداد تصویر بر حسب امتیاز ASPECT}                                                                                \\ \hline
     
    {شماره‌ی لایه} & تعداد تصویر کل & {۱۰} & {۹} & {۸} & {۷} & {۶} & ۵-۰ \\ \hline
    {1}                                    & 32                                     & {6}                          & {7}                         & {5}                         & {6}                         & {3}                         & 5   \\ 
    {2}                                    & 31                                     & {6}                          & {6}                         & {6}                         & {5}                         & {3}                         & 5   \\ 
    {3}                                    & 34                                     & {7}                          & {7}                         & {6}                         & {5}                         & {4}                         & 5   \\ 
    {4}                                    & 31                                     & {6}                          & {7}                         & {6}                         & {5}                         & {2}                         & 5   \\ 
    {5}                                    & 31                                     & {7}                          & {6}                         & {6}                         & {5}                         & {2}                         & 5   \\ \hline
    \end{tabular}

\برچسب{جدول:test-folds}
\پایان{لوح}

همانطور که مشخص است، مجموع تعداد تصویر این لایه‌ها، ۱۵۹ تصویر است که برابر با تعداد کل تصاویر در دسترس در این پژوهش می‌باشد.
جدول \ref{جدول:train-folds} نیز
اطلاعات مربوط به مجموعه‌ی آموزشی هر لایه‌ی اعتبارسنجی را نمایش می‌دهد.

\شروع{لوح}[ht]

\vspace{1.5em}

\تنظیم‌ازوسط
\شرح[اطلاعات مجموعه‌ی آموزشی پنج لایه‌ی اعتبارسنجی]{حجم و ترکیب امتیازهای ASPECT مجموعه‌ی آموزشی در پنج لایه‌ی اعتبارسنجی.}

\begin{tabular}{cccccccc}
    \cline{3-8}
                                                               &                                        & \multicolumn{6}{c}{تعداد تصویر بر حسب امتیاز ASPECT}                                                                                                                                                                             \\ \hline
     
    {شماره‌ی لایه} & تعداد تصویر کل & {۱۰} & {۹} & {۸} & {۷} & {۶} & ۵-۰ \\ \hline
    {1}                                    & 127                                    & {26}                         & {26}                        & {24}                        & {20}                        & {11}                        & 20  \\
    {2}                                    & 128                                    & {26}                         & {27}                        & {23}                        & {21}                        & {11}                        & 20  \\ 
    {3}                                    & 125                                    & {25}                         & {26}                        & {23}                        & {21}                        & {10}                        & 20  \\ 
    {4}                                    & 128                                    & {26}                         & {26}                        & {23}                        & {21}                        & {12}                        & 20  \\ 
    {5}                                    & 128                                    & {25}                         & {27}                        & {23}                        & {21}                        & {12}                        & 20  \\ \hline
    \end{tabular}

\برچسب{جدول:train-folds}
\پایان{لوح}

همانطور که مشخص است، جمع تعداد تصاویر هر مجموعه‌ی آموزشی با مجموعه‌ی متناظرش برابر با ۱۵۹ بوده و کل مجموعه‌داده را پوشش می‌دهد.

\قسمت{تشخیص‌های انجام‌شده}
در هر یک از پنج لایه‌ی اعتبارسنجی، مدل بر روی مجموعه‌ی آموزشی، یادگیری خود را تکمیل می‌کند و بر بروی
 مجموعه‌ی کاملا دیده‌نشده، آزموده می‌شود.
 درواقع در هر دور، تمام داده‌های آزمایشی به مدل عرضه می‌شوند تا امتیاز ASPECT دو‌بخشی‌شان توسط مدل تشخیص داده‌شود.
تشخیص مثبت به معنای امتیاز $\geq 6$ و تشخیص منفی به معنای امتیاز $<6$ می‌باشد.

در تشخیص‌های انجام‌شده توسط مدل، چهار حالت متصور است.
حالت اول، 
مثبتِ‌صحیح\footnote{TP}
می‌باشد.
این حالت تشخیص مثبتی را نشان می‌دهد که برچسب واقعی‌اش نیز مثبت است.
حالت دوم، 
مثبتِ‌ناصحیح\footnote{FP}
است.
این حالت، تشخیص مثبتی را نشان می‌دهد که برچسب واقعی‌اش منفی بوده‌است.
دو معیار دیگر نیز 
منفیِ‌صحیح\footnote{TN}
و منفیِ‌ناصحیح\footnote{FN}
هستند که به ترتیب، نشخیص منفی با برچسب واقعی منفی و برچسب واقعی مثبت را نشان می‌دهند.

جدول \ref{جدول:fold-diagnosis}، اطلاعات تشخیصی مدل در
هر لایه‌ی اعتبارسنجی و جدول \ref{جدول:overall-diagnosis}،
تشخیص‌های مدل بر روی کل داده‌ها را نمایش می‌دهد.
این مقادیر برای محاسبه‌ی معیار‌های ارزیابی مدل در قسمت بعد کاربرد دارند.

\شروع{لوح}[ht]

\vspace{1.5em}

\تنظیم‌ازوسط
\شرح[تشخیص‌های انجام‌شده توسط مدل در پنج لایه‌ی اعتبارسنجی]{تشخیص‌های انجام‌شده توسط مدل در پنج لایه‌ی اعتبارسنجی. ،TP ،FP ،TN ،FN ،P ،N ،T ،F و کل، به ترتیب تعداد تشخیص‌های مثبتِ‌صحیح، مثبتِ‌ناصحیح، منفیِ‌صحیح، منفیِ‌ناصحیح، تعداد داده‌های مثبت، تعداد داده‌های منفی، تعداد تشخیص‌های مثبت، تعداد تشخیص‌های منفی و تعداد کل داده‌ها را نشان می‌دهد.}

\begin{tabular}{cccccccccc}
    \hline
     
    شماره‌ی لایه & TP & FP & TN & FN & P & N  & T  & F & کل\\ \hline
    1           & 5  & 0  & 27 & 0  & 5 & 27 & 32 & 0 & 32 \\ 
    2           & 5  & 2  & 24 & 0  & 5 & 26 & 29 & 2 & 31\\ 
    3           & 3  & 5  & 24 & 2  & 5 & 29 & 27 & 7 & 34\\ 
    4           & 5  & 1  & 25 & 0  & 5 & 26 & 30 & 1 & 31\\ 
    5           & 5  & 3  & 23 & 0  & 5 & 26 & 28 & 3 & 31\\ \hline
    \end{tabular}

\برچسب{جدول:fold-diagnosis}
\پایان{لوح}

\شروع{لوح}[ht]

\vspace{1.5em}

\تنظیم‌ازوسط
\شرح[تشخیص‌های نهایی انجام‌شده توسط مدل]{تشخیص‌های نهایی انجام‌شده توسط مدل. ،TP ،FP ،TN ،FN ،P ،N ،T ،F و کل، به ترتیب تعداد تشخیص‌های مثبتِ‌صحیح، مثبتِ‌ناصحیح، منفیِ‌صحیح، منفیِ‌ناصحیح، تعداد داده‌های مثبت، تعداد داده‌های منفی، تعداد تشخیص‌های مثبت، تعداد تشخیص‌های منفی و تعداد کل داده‌ها را نشان می‌دهد.  }

\begin{tabular}{ccccccccc}
    \hline
     
    TP & FP & TN & FN  & P  & N     & T & F & کل  \\ \hline
    23 & 11 & 123& 2   & 25 & 134   &146& 13& 159\\ \hline
    \end{tabular}

\برچسب{جدول:overall-diagnosis}
\پایان{لوح}


\قسمت{معیار‌های ارزیابی}

مدل پیشنهادی در این پروژه بر اساس تعدادی از معیار‌های پرکاربرد یادگیری ماشین  
در حوزه‌ی طبقه‌بندی تصاویر 
ارزیابی می‌شود.
این معیارها عبارتند از
 دقت،\footnote{Accuracy}
حساسیت،\footnote{Sensitivity}
تشخیص،\footnote{Specificity}
صحت،\footnote{Precision}
بازیابی\footnote{Recall}
و 
مساحت زیر نمودار مشخصه‌ی عملیاتی گیرنده\footnote{AUC}.
جدول \ref{جدول:fold-results}
این معیار‌ها را در هر لایه‌ی اعتبارسنجی و جدول 
\ref{جدول:overall-results}
مقدار نهایی این معیار‌ها برای مدل را نمایش می‌دهد.
این مقادیر نهایی از اجتماع تمام پیش‌بینی‌های انجام شده توسط مدل بر روی تمام داده‌ها و مقایسه‌ی آن‌ها با برچسب‌های واقعی متناظر به‌دست آمده‌اند.

\شروع{لوح}[ht]

\vspace{1.5em}


\تنظیم‌ازوسط
\شرح[معیارهای ارزیابی مدل در پنج لایه‌ی اعتبارسنجی]{معیارهای ارزیابی مدل در پنج لایه‌ی اعتبارسنجی.}

\begin{tabular}{ccccccc}
    \hline
     
    شماره‌ی لایه & دقت     & حساسیت & تشخیص    & صحت     & بازیابی & AUC \\ \hline
    1           & 1       & 1      & 1         & 1       & 1        & 1   \\ 
    2           & 935.0   & 1      & 923.0     & 714.0   & 1        & 99.0   \\ 
    3           & 794.0   & 6.0    & 827.0     & 375.0   & 6.0      & 88.0   \\ 
    4           & 967.0   & 1      & 961.0     & 833.0   & 1        & 99.0   \\ 
    5           & 903.0   & 1      & 884.0     & 625.0   & 1        & 99.0   \\ \hline
    \end{tabular}

\برچسب{جدول:fold-results}
\پایان{لوح}

\شروع{لوح}[ht]

\vspace{1.5em}

\تنظیم‌ازوسط
\شرح[معیارهای ارزیابی نهایی مدل]{معیارهای ارزیابی نهایی مدل.}

\begin{tabular}{cccccc}
    \hline
     
    دقت     & حساسیت & تشخیص & صحت     & بازیابی   & AUC \\ \hline
    918.0   & 92.0    & 917.0 & 676.0   & 92.0       & 96.0   \\ \hline
    \end{tabular}

\برچسب{جدول:overall-results}
\پایان{لوح}

یکی از معیار‌های مورد استفاده در اعتبارسنجی مدل، AUC می‌باشد.
AUC یکی از مهم‌ترین معیارهای ارزیابی مدل‌های یادگیری ماشین طبقه‌بندی‌کننده است و
نشان می‌دهد مدل چقد توانایی تشخیص کلاس‌ها از هم را دارد.
هر چه این مدل به عدد ۱ نزدیک‌تر باشد، توانایی تشخیص هر کلاس به عنوان همان کلاس در مدل بیشتر است.
AUC که کوتاه‌شده‌ی \lr{Area Under Curve} است، درواقع مساحت زیر نمودار مشخصه‌ی عملیاتی گیرنده\footnote{ROC} می‌باشد.
نمودار AUC-ROC مدل در مجموع تمام داده‌ها در تصویر \ref{fig:auc-roc} آمده‌است.

\شروع{شکل}[ht]
\centerimg{auc-roc}{0.5\textwidth}
\caption[نمودار AUC-ROC مدل]{نمودار AUC-ROC مدل}
\برچسب{fig:auc-roc}
\پایان{شکل}

AUC مدل پیشنهادی در این پژوهش در مقایسه با AUC گزارش‌شده در تمام کار‌های پیشین، بیشترین مقدار را دارد و نشان از قدرت تشخیصی بالای مدل می‌باشد.