
\فصل{آزمایش‌ها}

در این فصل، پس از شرح روش آموزش و آزمایش مدل، نتایج به‌دست‌آمده از ارزیابی مدل ذکر می‌شود.
این نتایج به تفکیک هر لایه‌ی ارزیابی گزارش می‌شوند و جمع‌بندی عملکرد مدل بر روی کل مجموعه‌داده نیز عنوان می‌شود.


\قسمت{آموزش مدل}
در این قسمت، جزئیات فنی آموزش مدل شرح داده می‌شود.
تنظیماتی که در این قسمت ذکر می‌شوند نیز از یک جستجوی ساخت‌یافته بر روی حالت‌های ممکن و ارزیابی نتایج هر حالت به‌دست آمده‌است.
این تنظیمات و پارامترها شامل 
نوع تابع زیان،\footnote{\lr{Loss Function}}
نوع بهینه‌ساز،\footnote{Optimizer}
،نرخ یادگیری\footnote{\lr{Learning Rate}}
نوع و پارامتر‌های زمان‌بند،\footnote{Scheduler}
تعداد دور‌های آموزش،\footnote{Epoch}
اندازه‌ی دسته\footnote{\lr{Batch Size}}
و 
سیاست انتخاب مدل می‌باشند که مقدار آن‌ها به شرح زیر است.

\begin{itemize}
    \item \textbf{تابع زیان:} با توجه به چند-کلاسه بودن خروجی مدل پیشنهادی، تابع زیان مورد استفاده برای آموزش مدل، تابع زیان آنتروپی متقاطع
    \footnote{Crossentropy}
    چند-کلاسه می‌باشد.
    \item \textbf{الگوریتم بهینه‌سازی:} الگوریتم بهینه‌سازی‌ای که در آموزش مدل به‌کار گرفته‌شده، الگوریتم Adam است.
    لازم به ذکر است که روش مورد استفاده در این پروژه، یادگیری انتقالی است و نه تنظیم دقیق\footnote{\lr{Fine Tuning}}. بنابراین تمام پارامتر‌های مدل \lr{EfficientNetB0} ثابت می‌شوند و بهینه‌سازی، آن‌ها را دستخوش تغییر نمی‌کند.
    \item \textbf{نرخ یادگیری:} نرخ یادگیری در نظر گرفته‌شده برای بهینه‌سازی، $0.0005$ می‌باشد.
    \item \textbf{الگوریتم زمان‌بند:} الگوریتم مورد استفاده برای زمان‌بند، الگوریتم StepLR می‌باشد که در هر چند دور از آموزش، نرخ یادگیری را با یک ضریب کاهش می‌دهد. پارامتر‌های تنظیم‌شده برای این زمان‌بند به صورت زیر هستند.
    \begin{itemize}
        \item تعداد گام: 8
        \item ضریب کاهش :(Gamma) $0.1$
    \end{itemize}
    \item \textbf{تعداد دور‌های آموزش:} تعداد ۵۰ دور برای آموزش مدل در نظر گرفته‌شده‌است.
    \item \textbf{اندازه‌ی دسته:} تصاویر در دسته‌های 36تایی از مجموعه‌داده واکشی می‌شوند و در بهینه‌سازی شرکت می‌کنند.
    \item \textbf{سیاست انتخاب مدل:} در بسیاری از پروژه‌های یادگیری ماشین، بخشی از داده‌ها به عنوان مجموعه‌ی ارزیابی جدا می‌شوند و برای سنجش عملکرد مدل بر روی داده‌های دیده‌نشده در طی دور‌های آموزشی مورد استفاده قرار می‌گیرند.
    به این ترتیب، مدل در وضعیتی که بهترین نتایج را کسب کرده است، انتخاب شده و وارد فاز آزمایشی می‌شود.
    اما به علت محدودیت تعداد داده‌های این پژوهش، امکان جداسازی بخشی از داده‌ها به این منظور وجود ندارد.
    بنابراین سیاست انتخاب مدل باید تنها مبتنی بر عملکرد آن بر روی داده‌های آموزشی باشد.
    در این پروژه، مدلی که در یک دور آموزشی، بهترین عملکرد را در تشخیص دوبخشی امتیاز ASPECTS داده‌های آموزشی از خود نشان داده‌باشد، به عنوان مدل نهایی برگزیده می‌شود.
    نحوه‌ی ارزیابی مدل پیشنهادی که به روش ارزیابی متقابل است، این اطمینان را می‌دهد که مدل انتخاب‌شده به این روش، قابل اعتماد است و عملکرد خوبش بر روی داده‌های دیده‌شده را بر روی داده‌های دیده‌نشده نیز حفظ می‌کند.
\end{itemize}


\قسمت{آزمایش مدل}

با مقدمه‌ای که در بخش مفاهیم اولیه در رابطه با اعتبارسنجی متقابل آمد، در روش پیشنهادی این پروژه از ایده‌ای مشابه 
اعتبارسنجی متقابل ۵ لایه\footnote{\lr{5-fold Cross Validation}}
استفاده می‌شود.
به این معنا که مدل، ۵ مرتبه فرایند آموزش و آزمایش را طی می‌کند.
 در هر کدام از این دفعات، تقریباً یک‌پنجم داده‌ها در مجموعه‌ی دیده‌نشده برای آزمایش قرار می‌گیرند.
 تا اینکه نهایتاً پس از این ۵ مرحله، عملکرد مدل بر روی تمام داده‌های موجود ارزیابی شده‌باشد.
 در انتها، پس از کسب اطمینان از عملکرد مدل به این روش، مدل بر روی تمام داده‌های موجود آموزش می‌بیند و به عنوان مدل نهایی ارائه می‌شود.
 شکل \ref{fig:cross-val}
 شمایی از ارزیابی متقابل مورد استفاده در این پژوهش را نشان می‌دهد.

\شروع{شکل}[ht]
\vspace{0.5cm}

\centerimg{cross-val}{\textwidth}
\شرح[شمایی از روش ارزیابی متقابل]{شمایی از روش ارزیابی متقابل. مدل در ۵ مرحله‌ی مجزا مورد آموزش و آزمایش قرار می‌گیرد تا نهایتاً بر روی تمام داده‌ها ارزیابی شده‌باشد. نتایج مدل از تجمیع مجموعه‌های آزمایشی محاسبه می‌شوند و مدل نهایی نیز از آموزش بر روی تمام داده‌ها ارائه می‌شود.}
\برچسب{fig:cross-val}

\vspace{0.5cm}
\پایان{شکل}

در انتخاب مجموعه‌ی آزمایشی در هر لایه، از هر دسته امتیاز، به طور متوازنی داده‌ی آزمایشی انتخاب شده‌است.
به عنوان مثال اگر ۲۵ تصویر در یک کلاس وجود داشته، تقریباً ۵ تصویر از آن کلاس برای هر لایه انتخاب شده‌است.
همچنین به این نکته توجه شده‌است که تصاویر موجود برای هر بیمار، یا تماماً در دسته‌ی آموزشی قرار بگیرند و یا تماماً در مجموعه‌ی آزمایشی باشند.
به این ترتیب، جدایی و  دیده‌نشده‌بودن داده‌ها در ارزیابی مدل به طور کامل رعایت شده‌است.

لازم به ذکر است
روش پیشنهادی، مدل را به صورت چند-کلاسه آموزش می‌دهد اما سیاست انتخاب مدل و ارزیابی مدل هر دو مبنی بر طبقه‌بندی دو‌بخشی هستند.
به این معنا که خروجی چند-کلاسه‌ی مدل، 
در یکی از دو دسته‌ی 
$\geq 6$ باشد یا $< 6$
قرار می‌گیرد و با امتیاز واقعی مقایسه می‌شود.
نتایج گزارش شده به عنوان نتایج آزمایشی مدل نیز از همین نوع هستند. 

همانطور که ذکر شد،
افراز ۵ لایه‌ای انجام‌شده بر روی مجموعه‌داده، تقریباً متوازن است.
به این معنا که از هر امتیاز ASPECTS تقریباً تعداد یکسانی در هر قسمت وجود دارد.
به این ترتیب مدل در فرایند آموزش هر یک از ۵ لایه، تعداد نمونه‌ی مناسبی از هر امتیاز را مشاهده کرده و فرامی‌گیرد.
جدول \ref{جدول:test-folds} اندازه‌ی مجموعه‌ی آموزشی هر لایه و توزیع امتیازات ASPECTS در آن‌ها را نشان می‌دهد.

\شروع{لوح}[ht]

\vspace{1.5em}

\تنظیم‌ازوسط
\شرح[اطلاعات مجموعه‌ی آزمایشی پنج لایه‌ی ارزیابی]{حجم و ترکیب امتیازهای ASPECTS مجموعه‌ی آزمایشی در پنج لایه‌ی ارزیابی.}

\begin{tabular}{cccccccc}
    \cline{3-8}
                                                               &                                        & \multicolumn{6}{c}{تعداد تصویر بر حسب امتیاز ASPECTS}                                                                                \\ \hline
     
    {شماره‌ی لایه} & تعداد تصویر کل & {۱۰} & {۹} & {۸} & {۷} & {۶} & ۵-۰ \\ \hline
    {1}                                    & 32                                     & {6}                          & {7}                         & {5}                         & {6}                         & {3}                         & 5   \\ 
    {2}                                    & 31                                     & {6}                          & {6}                         & {6}                         & {5}                         & {3}                         & 5   \\ 
    {3}                                    & 34                                     & {7}                          & {7}                         & {6}                         & {5}                         & {4}                         & 5   \\ 
    {4}                                    & 31                                     & {6}                          & {7}                         & {6}                         & {5}                         & {2}                         & 5   \\ 
    {5}                                    & 31                                     & {7}                          & {6}                         & {6}                         & {5}                         & {2}                         & 5   \\ \hline
    \end{tabular}

\برچسب{جدول:test-folds}
\پایان{لوح}

همانطور که مشخص است، مجموع تعداد تصویر این لایه‌ها، ۱۵۹ تصویر است که برابر با تعداد کل تصاویر در دسترس در این پژوهش می‌باشد.
جدول \ref{جدول:train-folds} نیز
اطلاعات مربوط به مجموعه‌ی آموزشی هر لایه‌ی ارزیابی را نمایش می‌دهد.

\شروع{لوح}[ht]

\vspace{1.5em}

\تنظیم‌ازوسط
\شرح[اطلاعات مجموعه‌ی آموزشی پنج لایه‌ی ارزیابی]{حجم و ترکیب امتیازهای ASPECTS مجموعه‌ی آموزشی در پنج لایه‌ی ارزیابی.}

\begin{tabular}{cccccccc}
    \cline{3-8}
                                                               &                                        & \multicolumn{6}{c}{تعداد تصویر بر حسب امتیاز ASPECTS}                                                                                                                                                                             \\ \hline
     
    {شماره‌ی لایه} & تعداد تصویر کل & {۱۰} & {۹} & {۸} & {۷} & {۶} & ۵-۰ \\ \hline
    {1}                                    & 127                                    & {26}                         & {26}                        & {24}                        & {20}                        & {11}                        & 20  \\
    {2}                                    & 128                                    & {26}                         & {27}                        & {23}                        & {21}                        & {11}                        & 20  \\ 
    {3}                                    & 125                                    & {25}                         & {26}                        & {23}                        & {21}                        & {10}                        & 20  \\ 
    {4}                                    & 128                                    & {26}                         & {26}                        & {23}                        & {21}                        & {12}                        & 20  \\ 
    {5}                                    & 128                                    & {25}                         & {27}                        & {23}                        & {21}                        & {12}                        & 20  \\ \hline
    \end{tabular}

\برچسب{جدول:train-folds}
\پایان{لوح}

همانطور که مشخص است، جمع تعداد تصاویر هر مجموعه‌ی آموزشی با مجموعه‌ی متناظرش برابر با ۱۵۹ بوده و کل مجموعه‌داده را پوشش می‌دهد.

\قسمت{تشخیص‌های انجام‌شده}
در هر یک از پنج لایه‌ی ارزیابی، مدل بر روی مجموعه‌ی آموزشی، یادگیری خود را تکمیل می‌کند و بر بروی
 مجموعه‌ی کاملاً دیده‌نشده، آزموده می‌شود.
 درواقع در هر دور، تمام داده‌های آزمایشی به مدل عرضه می‌شوند تا امتیاز ASPECTS دو‌بخشی‌شان توسط مدل تشخیص داده‌شود.
تشخیص منفی به معنای امتیاز $\geq 6$ و تشخیص مثبت به معنای امتیاز $<6$ می‌باشد.

در تشخیص‌های انجام‌شده توسط مدل، چهار حالت متصور است.
حالت اول، 
مثبتِ‌صحیح\footnote{TP}
می‌باشد.
این حالت تشخیص مثبتی را نشان می‌دهد که برچسب واقعی‌اش نیز مثبت است.
حالت دوم، 
مثبتِ‌ناصحیح\footnote{FP}
است.
این حالت، تشخیص مثبتی را نشان می‌دهد که برچسب واقعی‌اش منفی بوده‌است.
دو معیار دیگر نیز 
منفیِ‌صحیح\footnote{TN}
و منفیِ‌ناصحیح\footnote{FN}
هستند که به ترتیب، تشخیص منفی با برچسب واقعی منفی و برچسب واقعی مثبت را نشان می‌دهند.

جدول \ref{جدول:fold-diagnosis}، اطلاعات تشخیصی مدل در
هر لایه‌ی ارزیابی و جدول \ref{جدول:overall-diagnosis}،
تشخیص‌های مدل بر روی کل داده‌ها را نمایش می‌دهد.
این مقادیر برای محاسبه‌ی معیار‌های ارزیابی مدل در قسمت بعد کاربرد دارند.

\شروع{لوح}[ht]

\vspace{1.5em}

\تنظیم‌ازوسط
\شرح[تشخیص‌های انجام‌شده توسط مدل در پنج لایه‌ی ارزیابی]{تشخیص‌های انجام‌شده توسط مدل در پنج لایه‌ی ارزیابی. ،TP ،FP ،TN ،FN ،P ،N ،T ،F و کل، به ترتیب تعداد تشخیص‌های مثبتِ‌صحیح، مثبتِ‌ناصحیح، منفیِ‌صحیح، منفیِ‌ناصحیح، تعداد داده‌های مثبت، تعداد داده‌های منفی، تعداد تشخیص‌های مثبت، تعداد تشخیص‌های منفی و تعداد کل داده‌ها را نشان می‌دهد.}

\begin{tabular}{cccccccccc}
    \hline
     
    شماره‌ی لایه & TP & FP & TN & FN & P & N  & T  & F & کل\\ \hline
    1           & 5  & 0  & 27 & 0  & 5 & 27 & 32 & 0 & 32 \\ 
    2           & 5  & 2  & 24 & 0  & 5 & 26 & 29 & 2 & 31\\ 
    3           & 3  & 5  & 24 & 2  & 5 & 29 & 27 & 7 & 34\\ 
    4           & 5  & 1  & 25 & 0  & 5 & 26 & 30 & 1 & 31\\ 
    5           & 5  & 3  & 23 & 0  & 5 & 26 & 28 & 3 & 31\\ \hline
    \end{tabular}

\برچسب{جدول:fold-diagnosis}
\پایان{لوح}

\شروع{لوح}[ht]

\vspace{1.5em}

\تنظیم‌ازوسط
\شرح[تشخیص‌های نهایی انجام‌شده توسط مدل]{تشخیص‌های نهایی انجام‌شده توسط مدل. ،TP ،FP ،TN ،FN ،P ،N ،T ،F و کل، به ترتیب تعداد تشخیص‌های مثبتِ‌صحیح، مثبتِ‌ناصحیح، منفیِ‌صحیح، منفیِ‌ناصحیح، تعداد داده‌های مثبت، تعداد داده‌های منفی، تعداد تشخیص‌های مثبت، تعداد تشخیص‌های منفی و تعداد کل داده‌ها را نشان می‌دهد.  }

\begin{tabular}{ccccccccc}
    \hline
     
    TP & FP & TN & FN  & P  & N     & T & F & کل  \\ \hline
    23 & 11 & 123& 2   & 25 & 134   &146& 13& 159\\ \hline
    \end{tabular}

\برچسب{جدول:overall-diagnosis}
\پایان{لوح}


\قسمت{نتایج ارزیابی}

جدول \ref{جدول:fold-results}
نتایج ارزیابی مدل در هر لایه‌ی آزمایشی و هر یک از معیار‌های
دقت،\footnote{Accuracy}
حساسیت،\footnote{Sensitivity}
تشخیص،\footnote{Specificity}
صحت،\footnote{Precision}
بازیابی\footnote{Recall}
و 
مساحت زیر نمودار مشخصه‌ی عملیاتی گیرنده\footnote{AUC}
را نشان می‌دهد.
جدول 
\ref{جدول:overall-results}
نیز
مقدار نهایی این معیار‌ها برای مدل را نمایش می‌دهد.
این مقادیر نهایی از اجتماع تمام پیش‌بینی‌های انجام شده توسط مدل بر روی تمام داده‌ها و مقایسه‌ی آن‌ها با برچسب‌های واقعی متناظر به‌دست آمده‌اند.

\شروع{لوح}[ht]

\vspace{1.5em}


\تنظیم‌ازوسط
\شرح[معیارهای ارزیابی مدل در پنج لایه‌ی ارزیابی]{معیارهای ارزیابی مدل در پنج لایه‌ی ارزیابی.}

\begin{tabular}{ccccccc}
    \hline
     
    شماره‌ی لایه & دقت     & حساسیت & تشخیص    & صحت     & بازیابی & AUC \\ \hline
    1           & 1       & 1      & 1         & 1       & 1        & 1   \\ 
    2           & 935.0   & 1      & 923.0     & 714.0   & 1        & 99.0   \\ 
    3           & 794.0   & 6.0    & 827.0     & 375.0   & 6.0      & 88.0   \\ 
    4           & 967.0   & 1      & 961.0     & 833.0   & 1        & 99.0   \\ 
    5           & 903.0   & 1      & 884.0     & 625.0   & 1        & 99.0   \\ \hline
    \end{tabular}

\برچسب{جدول:fold-results}
\پایان{لوح}

\شروع{لوح}[ht]

\vspace{1.5em}

\تنظیم‌ازوسط
\شرح[معیارهای ارزیابی نهایی مدل]{معیارهای ارزیابی نهایی مدل.}

\begin{tabular}{cccccc}
    \hline
     
    دقت     & حساسیت & تشخیص & صحت     & بازیابی   & AUC \\ \hline
    918.0   & 92.0    & 917.0 & 676.0   & 92.0       & 96.0   \\ \hline
    \end{tabular}

\برچسب{جدول:overall-results}
\پایان{لوح}

یکی از معیار‌های مورد استفاده در ارزیابی مدل، AUC می‌باشد.
AUC یکی از مهم‌ترین معیارهای ارزیابی مدل‌های یادگیری ماشین طبقه‌بندی‌کننده است و
نشان می‌دهد مدل چقد توانایی تشخیص کلاس‌ها از هم را دارد.
هر چه این مدل به عدد ۱ نزدیک‌تر باشد، توانایی تشخیص هر کلاس به عنوان همان کلاس در مدل بیشتر است.
AUC که کوتاه‌شده‌ی \lr{Area Under Curve} است، درواقع مساحت زیر نمودار مشخصه‌ی عملیاتی گیرنده\footnote{ROC} می‌باشد.
نمودار AUC-ROC مدل در مجموع تمام داده‌ها در تصویر \ref{fig:auc-roc} آمده‌است.

\شروع{شکل}[ht]
\centerimg{auc-roc}{0.5\textwidth}
\caption[نمودار AUC-ROC مدل]{نمودار AUC-ROC مدل}
\برچسب{fig:auc-roc}
\پایان{شکل}

AUC مدل پیشنهادی در این پژوهش در مقایسه با AUC گزارش‌شده در تمام کار‌های پیشین، بیشترین مقدار را دارد و نشان از قدرت تشخیصی بالای مدل می‌باشد.

\قسمت{مقایسه‌ی نتایج}

در فصل کارهای پیشین ذکر شد که در جستجوی تحقیقات حوزه‌ی ،ASPECTS
پژوهش معتبری با نوع برچسب و محدودیت مشابه این پروژه به‌دست نیامده‌است.
بنابراین مقایسه‌ی نتایج این پروژه با پژوهش‌های دیگر به اندازه‌ی کافی دقیق نخواهد بود.
با این حال،
نتایج به‌دست آمده نشان می‌دهد که
مدل پیشنهادی، علی‌رغم تمام محدودیت‌های موجود، در کلاس پژوهش‌هایی با مجموعه‌داده‌های بزرگ‌تر قرار می‌گیرد و
حتی در برخی معیارها از آن‌ها پیشی گرفته‌است.

از جمله‌ی این پژوهش‌ها  
می‌توان به 
\cite{cao2022deep} اشاره کرد که 
پژوهش متن‌بازی بر روی بیش از هزار بیمار در سال ۲۰۲۲ بوده و دقت دوبخشی و AUC در حدود 91.0 گزارش کرده‌است.
نمونه‌ی دیگر، پژوهش تازه‌تری در سال ۲۰۲۳ با تصاویر بیش از ۳۰۰ بیمار می‌باشد که AUC امتیازدهی دوبخشی ($\leq 4$) را 89.0 اعلام کرده‌است \cite{lee2023clinical}.
علاوه بر این موارد، دو پژوهش دوبخشی دیگر با اطلاعات بیش از ۲۵۰ بیمار، در سال‌های ۲۰۲۲ و ۲۰۲۱، به ترتیب دقت
889.0 و
90.0
و
AUC
82.0 
و 
87.0 
را اعلام کرده‌اند \cite{chiang2022deep,kuang2021eis}.
یک پژوهش نیز با محدودیت داده‌ای تقریباً مشابه، دقتی در حدود 70.0 گزارش نموده‌است \cite{yu2021automated}.
در این بین، یک پژوهش قدیمی‌تر، مربوط به چند سال قبل وجود دارد که دقت 96.0 و AUC 89.0 را در بخش نتایج دو‌بخشی ($\leq 4$) ارائه کرده‌است \cite{kuang2019automated}.

پیش‌تر ذکر شد که دو مورد از کارهای پیشین که مستقیماً اشاره‌ای به روش و مجموعه‌داده‌ی مورد استفاده در ارزیابی خود نکرده‌اند \cite{naganuma2021alberta} و یا جدایی داده‌های آزمایشی را رعایت نکرده‌اند \cite{golkonda2022automated} در مقایسه‌ی نتایج ذکر نمی‌شوند.
جدول \ref{جدول:result-comparison}، خلاصه‌ای از نتایج به‌دست آمده در سایر کارهای پیشین بر روی طبقه‌بندی دو‌بخشی ASPECTS و نتایج پروژه‌ی حاضر را نمایش می‌دهد.

\شروع{لوح}[ht]

\vspace{1.5em}

\تنظیم‌ازوسط
\شرح[مقایسه‌ی نتایج با کارهای پیشین]{مقایسه‌ی نتایج با کارهای پیشین. در رابطه با خانه‌هایی که با علامت - مشخص شده‌اند، اطلاعاتی گزارش نشده‌است. ستون‌های جدول از راست به چپ، مرجع مورد مقایسه، حجم مجموعه‌داده‌ی موجود، سطح برچسب در دسترس، آستانه‌ی دوبخشی طبقه‌بندی ASPECTS و معیار‌های ارزیابی مورد مقایسه‌ی این مراجع را نشان می‌دهند.}

    \begin{tabular}{cccccccc}
    \hline
    مرجع                                    & داده‌ها           & برچسب                     & آستانه‌ & دقت & حساسیت & تشخیص & \textbf{AUC}  \\ \hline
    \multirow{2}{*}{\cite{lee2023clinical}} & \multirow{2}{*}{> 800} & \multirow{2}{*}{ناحیه} & $\geq$ 5              & -   & 940.0  & 619.0 & \textbf{89.0} \\ 
                                            &                       &                            & $\geq$ 7               & -   & 954.0  & 766.0 & \textbf{80.0} \\ 
    \cite{cao2022deep}                      & > 1000                &     پیکسل                 & $\geq$ 7               & 913.0 &955.0 & 866.0  & \textbf{911.0} \\ 
    \cite{chiang2022deep}                   & > 250                 &     ناحیه                 & $\geq$ 6               & 889.0 &722.0 & 907.0  & \textbf{82.0} \\ 
    \cite{kuang2019automated}               & > 250                 &     پیکسل                 & $\geq$ 5               & 960.0 &978.0 & 800.0  & \textbf{89.0} \\ 
    \cite{kuang2021eis}                     & > 250                 &     پیکسل                 & $\geq$ 5               & 900.0 &-    & -       & \textbf{87.0} \\
    \cite{yu2021automated}                  & 90                    & -                          & $\geq$ 8               & 75.0-68.0 &63.0-61.0  & 35.0-19.0 & \textbf{-} \\
    پژوهش حاضر                             & 100                   &     مغز                   & $\geq$ 6               & 918.0 &920.0    & 917.0  & \textbf{96.0} \\ \hline
    \end{tabular}

\برچسب{جدول:result-comparison}
\پایان{لوح}

\قسمت{جمع‌بندی}

مدل پیشنهادی این پروژه، پس از طراحی و بهینه‌سازی، مورد آزمون قرار گرفته‌است.
در این فصل، علاوه بر تشریح روش ارزیابی مدل، نتایج حاصل از این ارزیابی به صورت مدون ارائه شده و با کارهای پیشین مقایسه گشته‌است.
ارزیابی مدل و مقایسه‌ی آن با پژوهش‌های گذشته از آن جهت اهمیت دارد که جایگاه آن را مشخص می‌کند و 
با در نظر گرفتن محدودیت‌های موجود می‌تواند معیاری از موفقیت آن را نیز به دست دهد.
همانطور که در این فصل عنوان شد، روش پیشنهادی این مدل به نتایج رضایت‌بخشی رسیده‌است و در مقایسه با سایر مطالعات انجام‌شده،
به جایگاه برجسته‌ای دست یافته‌است.
با این حال، در مورد این پژوهش، کاستی‌ها و محدودیت‌هایی نیز وجود دارد که در فصل آتی تشریح خواهد شد.

