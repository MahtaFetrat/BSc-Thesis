
% -------------------------------------------------------
%  Abstract
% -------------------------------------------------------


\شروع{وسط‌چین}
\مهم{چکیده}
\پایان{وسط‌چین}
\بدون‌تورفتگی

سکته‌ی مغزی به عنوان دومین عامل مرگ‌و‌میر در جهان شناخته می‌شود.
این عارضه می‌تواند آسیب‌های دائمی و جبران‌ناپذیری برای افراد مبتلا به همراه داشته‌باشد \cite{donkor2018stroke}.
بنابراین،
تشخیص سریع سکته‌ی مغزی و درمان در مراحل اولیه، از اهمیت بسیار بالایی برخوردار است.
امتیاز ASPECT یک معیار برای ارزیابی وخامت سکته‌ی مغزی بر روی تصاویر CT می‌باشد.
اما تشخیص صحیح وخامت سکته بر روی تصاویر اولیه، که نواحی آسیب‌دیده به خوبی در آن ظاهر نمی‌شود، حتی برای متخصصین این حوزه، امری چالش‌بر‌انگیز است.
 در این پژوهش 
یک راهکار تشخیص خودکار امتیاز ،ASPECT مبتنی بر روش‌های یادگیری ژرف ارائه می‌شود که می‌تواند
 در تعیین وخامت سکته توسط متخصصین این امر راهگشا باشد و سرعت و دقت تشخیص و تجویز روش‌های درمانی را بهبود ببخشد.
پژوهش حاضر، یک روش نظام‌مند برای پیش‌پردازش تصاویر CT مغزی پیشنهاد می‌کند که می‌تواند در سایر پژوهش‌ها بر روی تصاویر مغزی مورد استفاده قرار بگیرد.
در ادامه، یک شبکه‌ی عصبی عمیق، مبنی بر مدل‌های پیش‌آموزش‌یافته طراحی می‌شود و 
عملکرد این شبکه بر روی داده‌های موجود ارزیابی می‌گردد
که نتایج رضایت‌بخشی را را نشان می دهد.

\پرش‌بلند
\بدون‌تورفتگی \مهم{کلیدواژه‌ها}: 
سکته‌ی مغزی، ،ASPECTS یادگیری ژرف، ،CT یادگیری انتقالی
\صفحه‌جدید
