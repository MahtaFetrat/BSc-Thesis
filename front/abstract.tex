
% -------------------------------------------------------
%  Abstract
% -------------------------------------------------------


\شروع{وسط‌چین}
\مهم{چکیده}
\پایان{وسط‌چین}
\بدون‌تورفتگی

سکته‌ی مغزی به عنوان دومین عامل مرگ‌و‌میر در جهان شناخته می‌شود.
این عارضه می‌تواند آسیب‌های دائمی و جبران‌ناپذیری برای افراد مبتلا به همراه داشته‌باشد.
بنابراین،
تشخیص سریع سکته‌ی مغزی و درمان در مراحل اولیه، از اهمیت بسیار بالایی برخوردار است.
امتیاز ASPECTS یک معیار برای ارزیابی وخامت سکته‌ی مغزی از روی تصاویر سی‌تی‌اسکن مغز می‌باشد.
اما تشخیص صحیح این امتیاز از روی تصاویر اولیه، که نواحی آسیب‌دیده به خوبی در آن ظاهر نمی‌شود، حتی برای متخصصین این حوزه، امری چالش‌بر‌انگیز است.
به همین دلیل، در سال‌های اخیر، پژوهش‌های زیادی بر روی تشخیص خودکار امتیاز ASPECTS به انجام رسیده‌است.
روش‌های خودکار امتیاز‌دهی ASPECTS می‌توانند
 در تعیین وخامت سکته توسط متخصصین مورد استفاده قرار بگیرند و سرعت و دقت تشخیص و انتخاب روش‌های درمانی را افزایش دهند.
 در پژوهش حاضر، 
 یک راهکار نظام‌مند برای تشخیص امتیاز ASPECTS مبتنی بر روش‌های یادگیری ژرف ارائه شد‌ه‌است.
 روش پیشنهادی این پروژه،
 به دقت ۹۱۸.۰ و AUC ۹۶.۰ دست یافته‌است که
 در مقایسه با سایر پژوهش‌های مشابه، عملکرد برجسته‌ای را نشان می‌دهد و
می‌تواند در مطالعات آتی راهگشا باشد.

\پرش‌بلند
\بدون‌تورفتگی \مهم{کلیدواژه‌ها}: 
سکته‌ی مغزی، ،ASPECTS یادگیری ژرف، ،CT یادگیری انتقالی
\صفحه‌جدید
